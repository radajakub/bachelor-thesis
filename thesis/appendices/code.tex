\documentclass[../main.tex]{subfiles}

\begin{document}
\chapter{Implementation details}\label{apx:impl}
In this appendix, the implementation details for both the bandit iteration and the B-HSVI algorithms are shortly described with the focus on how to execute the respective methods.

\section{Bandit iteration}
The implementation is located in a git repository on \url{https://gitlab.fel.cvut.cz/radajak5/HSVIforPOPOSG.git}.
Even though, the repository is named after the subclass of partially observable stochastic games with public observations, this functionality is not ready yet.
Instead, it offers solving methods for stochastic games for the first experimental evaluation, the value iteration \refsec{standard:sg:valueiter:algorithm} as the reference method and bandit iteration \refsec{new:bandititeration} as the tested method.
Moreover, it provides means to load and generate the two SG instances, \textit{Tag} and \textit{Chase} \ref{exp:sg:games}.

To install this package, clone the repository, and then it can be imported to an environment with the standard \texttt{Pkg} julia commands.

Note, that for the sake of simplicity, only the functions essential for the game generators and the individual algorithms are presented.

\subsection{Instance generators}
An already created game can be loaded into a special \texttt{SG} structure, which can then be passed to the two algorithms, by calling its constructor with the \textit{path} to the file as the only parameter.
\begin{lstlisting}
sg = SG("./path/to/game/file.txt")
\end{lstlisting}

To generate an instance of the \textit{Tag} game, one needs to create a \texttt{map.txt} text file, which contains map of the maze similar to the games in \reffig{exp:sg:games:chase:examples} and \reffig{exp:sg:games:tag:examples} and to the following example.
\begin{lstlisting}
#####
# # #
# # #
#   #
#####
\end{lstlisting}
The stochastic game with this map is then generated by the following command.
\begin{lstlisting}
sg = generateSGtag("./path/to/map.txt")
\end{lstlisting}

To generate an instance of the \textit{Chase} game, a custom random graph generator is used with the following function.
\begin{lstlisting}
sg = generateSGrandom(states;
    outdegree = 1:1,
    multiplicity = 1:1,
    penalty = -1:-1,
    victory = 10:10,
    seed = -1
)
\end{lstlisting}
The parameters have the following effects
\begin{itemize}
    \item \texttt{states} $\dots$ the number of vertices of the generated graph,
    \item \texttt{outdegree} $\dots$ optional range argument to specify the number of edges going out from each vertex (the actual number is uniformly sampled from this range),
    \item \texttt{multiplicity} $\dots$ optional range argument to specify the number of potential targets of a single action, i.e. actions with stochastic effects,
    \item \texttt{penalty} $\dots$ optional range from which the rewards for a step without catching the opponent are uniformly sampled,
    \item \texttt{victory} $\dots$ the same as penalty but the rewards are for catching the opponent,
    \item \texttt{seed} $\dots$ optional initial seed for the pseudo-random generator
\end{itemize}

\subsection{Algorithms and parameters}
To execute the reference value iteration algorithm on a loaded \texttt{sg} instance with a precision $\epsilon > 0$, the following command is used.
\begin{lstlisting}
V = valueiteration(sg;
    gap::Real = 1e-6,
    discount::Real = -1,
    output::Type{<:Output} = Brief
)
\end{lstlisting}
The important parameters are \texttt{gap}, which is the maximal absolute value between the last two approximations of $V^*$ as described in \ref{standard:sg:valueiter}, \texttt{discount} which can change the default discount factor $\gamma = 0.95$ to another value.
The \texttt{output} parameter can be set to \texttt{Detailed} to return a vector of all approximations of V instead of only the last approximation in the case of the \texttt{Brief} option.

To execute the tested bandit iteration algorithm on a loaded \texttt{sg} instance serves the following command.
\begin{lstlisting}
V = bandititeration(sg::SG,
    bandittype::Type{<:Bandit};
    epsilon::Real = 0.0,
    expgamma::Real = 0.0,
    expeta::Real = 0.0,
    alpha::Real = 0.0,
    maxtrials::Integer = 0,
    iterations::Integer = 1000,
    seed::Integer = -1,
    discount::Real = -1,
    output::Type{<:Output} = Brief,
    step::Type{<:Step} = LinT
)
\end{lstlisting}
The \texttt{bandittype} is one of the tested bandit algorithms with its corresponding parameter, i.e. \texttt{(Observable)BestOfN} with \texttt{maxtrials}, \texttt{(Observable)EpsilonGreedy} with \texttt{epsilon}, \texttt{(Observable)SuccessiveElimination} and \texttt{(Observable)UCB} with \texttt{alpha} and \texttt{Exp3} with \texttt{expgamma} and \texttt{expeta}.
The other optional parameters have the following meanings:
\begin{itemize}
    \item \texttt{iterations} $\dots$ number of iterations of the run,
    \item \texttt{seed} $\dots$ initial value for the pseudo-random generator
    \item \texttt{discount} $\dots$ change the default discount factor $\gamma = 0.95$ to another value
    \item \texttt{output} $\dots$ the \texttt{Detailed} option returns a vector of all approximations, matrix of bandit algorithms for every state and the initial seed instead of the last approximation in the case of \texttt{Brief} option,
    \item \texttt{step} $\dots$ choose the accumulation step function, either \texttt{LinT} or \texttt{SqrtT}.
\end{itemize}

These versions of the algorithms were used to conduct the experiments as in \refsec{exp:sg}.
The other possible call options with non-essential parameters can be clearly understood from the code, but these listed options are sufficient to run the algorihtms.

\section{B-HSVI}
This B-HSVI algorithm continues the work of \cite{gitlab} in the git repository \url{https://gitlab.fel.cvut.cz/brozjak2/HSVIforOneSidedPOSGs.jl.git}.
The implemented bandit alternative is located in the \textit{bandits} git branch.
The core of the algorithm is their work, the contribution of this thesis are the bandit algorithms and the other modifications in the mentioned branch.

As the repository and possible parameters is very large, we describe only the parameters essential for our purposes, the rest can be viewed in the mentioned repository.

The instance used for evaluation is located in \textit{"HSVIforOneSidedPOSGs.jl/games/pursuit-evasion/no-fail/peg03.posg"}, but the other instances can be used as well.
After the package is installed according to the standard procedure, i.e. cloning the repository and adding the package to the desired environment, the only available function is \texttt{hsvi} in detail described below.
\begin{lstlisting}
hsvi("./path/to/game/file.posg", epsilon;
    ub_value_method::String = "lp",
    stage_game_method::String = "bandit",
    alpha_vector_creation::String = "lp",
    evaluate_strategy::Bool = false,
    seed::Int64 = 42,
    bandit_type::Type{<:Bandit} = EpsilonGreedy,
    bandit_epsilon::Float32 = 0.1f0,
    bandit_maxtrials::Int64 = 10,
    bandit_alpha::Float32 = 10f0,
    bandit_gamma::Float32 = 0.1f0,
    alpha_vector_epsilon::Float32 = 0f0,
    alpha_vector_random_size::Int64 = 10,
    experiments_dir::String = "",
    change_discount::Float32 = 0f0,
    change_partition::Int64 = 0,
    change_belief::Vector{Float32} = Vector{Float32}([]),
    max_explores::Int64 = 50000
) 
\end{lstlisting}
In the following list, only the parameters which can be \textit{changed} to test the bandit algorithm are shown, the other parameters are necessary to remain set to the written values.
\begin{itemize}
    \item \texttt{seed} $\dots$ the initial value for the pseudo-random generator,
    \item \texttt{bandit\_type} $\dots$ one of the following bandits with its respective parameter: \texttt{BestOfN} with \texttt{bandit\_maxtrials}, \texttt{EpsilonGreedy} with \texttt{bandit\_epsilon}, \texttt{SuccessiveElimination} and \texttt{UCB} with \texttt{bandit\_alpha} and \texttt{Exp3} with \texttt{bandit\_gamma},
    \item \texttt{alpha\_vector\_epsilon} $\dots$ probability of choosing $\alpha$-vector uniformly from the set of best $\alpha$-vectors of size \texttt{alpha\_vector\_random\_size}, instead of selecting always the best one,
    \item \texttt{experiments\_dir} $\dots$ directory for the output files of the course of search,
    \item \texttt{change\_discount} $\dots$ different value as a discount factor $\gamma$ than the default $0.95$,
    \item \texttt{change\_partition} $\dots$ different initial information set than the one specified in the game file,
    \item \texttt{change\_belief} $\dots$ different to default initial belief over the states of the specified information set,
    \item \texttt{max\_explores} $\dots$ maximal number of bandit updates of both value function bounds (the search is terminated after the closes return from the recursive call of EXPLORE procedure).
\end{itemize}
Note that the method to compute the value of the upper bound and creation of the new $\alpha$-vector is still using the linear programming method.
However, the stage game linear program was avoided by the use of the bandit algorithms.

This description of the execution of the B-HSVI algorithm is sufficient for our purposes.
For more detail, visit the repository page.

\end{document}